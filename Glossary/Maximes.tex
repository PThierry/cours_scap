%\usepackage{maximes}
%\addMaxime[author=Lewis Caroll{,} Alice{'}s Adventures in Wonderland, 
%fr=\textquoteMy{Voulez-vous me dire{,} s'il vous pla�t{,} quel chemin je dois prendre � partir d'ici ?}\maximeNewline \textquoteMy{Cela d�pend grandement de o� vous voulez aller}{,} dit le Chat.\maximeNewline \textquoteMy{Peut m'importe o�}{,} dit Alice.\maximeNewline \textquoteMy{Alors le chemin que vous prenez n'a pas d'importance.}{,} dit le Chat., 
%en=\textquoteMy{Would you tell me{,} please{,} which way I ought to go from here?}\maximeNewline \textquoteMy{That depends a good deal on where you want to get to}{,} said the Cat.\maximeNewline \textquoteMy{I don't much care where}{,} said Alice.\maximeNewline \textquoteMy{Then it doesn't matter which way you go}{,} said the Cat., 
%cite=, choose=en{,}fr, 
%ratio=0.85]

\addMaxime[author=Antoine de Saint-Exup�ry, 
fr=Et je n'ai point d'espoir de sortir par moi de ma solitude.\maximeNewline La pierre n'a point d'espoir d'�tre autre chose que pierre.\maximeNewline Mais{,} de collaborer{,} elle s'assemble\maximeNewline et devient Temple., 
en=I have no hope of getting out of my solitude by myself.\maximeNewline Stones have no hope of being anything but stones.\maximeNewline However{,} through collaboration they get themselves together\maximeNewline and become a Temple.,
cite=Exupery_Citadelle, ratio=0.86]

\addMaxime[author=Antoine de Saint-Exup�ry, 
fr=�tre homme{,} c'est pr�cis�ment �tre responsable.\maximeNewline C'est conna�tre la honte face � une mis�re qui ne semblait pas d�pendre de soi.\maximeNewline C'est �tre fier d'une victoire que les camarades ont remport�e.\maximeNewline C'est sentir{,} en posant sa pierre{,}\maximeNewline que l'on contribue � b�tir le monde., 
en=To be a man is{,} precisely{,} to be responsible.\maximeNewline It is to feel shame at the sight of what seems to be unmerited misery.\maximeNewline It is to take pride in a victory won by one's comrades.\maximeNewline It is to feel{,} when setting one's stone{,}\maximeNewline that one is contributing to the building of the world.,
cite=Exupery_Terre, ratio=0.86]

%""
   

\addMaxime[author=Lewis Caroll{,} Alice{'}s Adventures in Wonderland, 
fr=\textquoteMy{Would you tell me{,} please{,} which way I ought to go from here?}\maximeNewline \textquoteMy{That depends a good deal on where you want to get to}{,} said the Cat.\maximeNewline \textquoteMy{I don't much care where}{,} said Alice.\maximeNewline \textquoteMy{Then it doesn't matter which way you go}{,} said the Cat., 
cite=, choose=fr, 
ratio=0.86]

\addMaxime[author=Lewis Caroll{,} Alice au Pays des Merveilles, 
fr=\textquoteMy{Voulez-vous me dire{,} s'il vous pla�t{,} quel chemin je dois prendre � partir d'ici ?}\maximeNewline \textquoteMy{Cela d�pend grandement de o� vous voulez aller}{,} dit le Chat.\maximeNewline \textquoteMy{Peut m'importe o�}{,} dit Alice.\maximeNewline \textquoteMy{Alors le chemin que vous prenez n'a pas d'importance.}{,} dit le Chat., 
cite=, choose=fr, 
ratio=0.86]

\addMaxime[author=Charles Darwin's brother Erasmus, fr=Finalement{,} le raisonnement � priori est si satisfaisant pour moi que si\maximeNewline les faits ne correspondent pas{,} mon sentiment est : tant pis pour les faits., en=In fact the � priori reasoning is so entirely satisfactory to me that if\maximeNewline the facts won't fit in{,} why so much the worse for the facts is my feeling., cite=Erasmus_Darwin, ratio=0.86]

\addMaxime[author=Cioran, 
fr=N'a de convictions que celui qui n'a rien approfondi., 
en=We have convictions only if we have studied nothing thoroughly., 
cite=Emile_Cioran, ratio=0.8]

%\addMaxime[author=Albert Einstein, fr=On ne r�sout pas les probl�mes avec les modes de pens�e qui les ont engendr�s., en=We can't solve problems by using the same kind of thinking we used when we created them., cite=Albert_Einstein, ratio=0.5]

\addMaxime[author=Alexandre Jollien, 
fr=Troisi�me principe pour rester zen{,} le principe de Yunmen : \textquoteMy{Quand tu marches{,} marche{,} quand tu es assis{,} sois assis. Surtout{,} n'h�sites pas.} L'autre jour{,} aux toilettes{,} je me suis surpris en train de me brosser les dents tout en r�pondant au t�l�phone. Selon le principe de Yunmen{,} il y avait au moins deux choses en trop., 
en=Third principle to remain zen{,} the principle of Yunmen: \textquoteMy{When you walk{,} walk{,} when you sit{,} be seated. Above all{,} do not hesitate.} The other day{,} in the bathroom{,} I surprise myself by brushing my teeth while answering the phone. According to the principle of Yunmen{,} there were at least two things too many.,
cite=Alexandre_Jollien, ratio=0.86]




\addMaxime[author=Karl Von Frisch, 
fr=La fourmi est un animal intelligent collectivement et stupide individuellement ;\maximeNewline l'homme c'est l'inverse., 
en=The ant is a collectively intelligent and individually stupid animal;\maximeNewline man is the opposite., 
ratio=0.86]

\addMaxime[author=John F. Woods, 
fr=Codez toujours en pensant que celui qui maintiendra votre code est un psychopathe qui connait votre adresse., 
en=Always code as if the guy who ends up maintaining your code will be a violent psychopath who knows where you live., 
cite=John_Woods, ratio=0.86]

\addMaxime[author=Hans Henrik Bohr, 
fr=Une des maximes favorites de mon p�re �tait la distinction entre les deux sortes de v�rit�s{,} des v�rit�s profondes reconnues par le fait que l'inverse est �galement une v�rit� profonde{,} contrairement aux banalit�s o� les contraires sont clairement absurdes., 
en=One of the favorite maxims of my father was the distinction between the two sorts of truths{,} profound truths recognized by the fact that the opposite is also a profound truth{,} in contrast to trivialities where opposites are obviously absurd., cite=Hans_Bohr, ratio=0.86]



%\addMaxime[author=Cioran, fr=La conscience est bien plus que l'�charde{,}\maximeNewline elle est le poignard dans la chair., en=Consciousness is much more than the thorn{,}\maximeNewline it is the dagger in the flesh., ratio=0.5]

%Aujourd'hui, la censure a chang� de visage. Ce n'est plus le manque qui agit mais l'abondance.

%There is an age-old adage, 'If the facts don't fit the theory, change the theory.' But too often it's easier to keep the theory and change the facts.
%if the facts won't fit, then so much worse for the facts
%In fact the � priori reasoning is so entirely satisfactory to me that if the facts won't fit in, why so mush the worse for the facts is my feeling. , Erasmus Darwin, Charles Darwin's brother Erasmus %http://books.google.fr/books?id=v3Pb8og_zvMC&lpg=PP1&pg=PA228&redir_esc=y#v=onepage&q&f=false

