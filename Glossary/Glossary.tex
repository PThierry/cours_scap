% Pour compiler les glossaires et les acronymes : 
% 1. Compiler PDLatex/Latex
% 2. Compiler les glossaires : makeglossaries %
%
%
% Pour compiler uniquement le glossaire
% 1. Compiler avec PDFLaTex/Latex
% 2. Compiler le glossaire : makeindex -s  %.ist -t $1.glg -o $1.gls $1.glo
% 
%
% Pour compiler uniquement les acronymes
% 1. Compiler avec PDFLaTex/Latex
% 2. Compiler les acronymes : makeindex -s %.ist -t %.alg -o %.acr %.acn
% 

%% USE : 
%% \addAcronymAndGlossary{key}{name}{fullname}{description}; 
%% Then call \gls{key}
%% First time, we will display : "fullname (name)".
%% Then : "name"
%%
%% If description is not empty, a glossary entry is also created with the description.


%\addAcronymAndGlossary{api}{API}{}{\textbf{A}pplication \textbf{P}rogramming \textbf{I}nterface}{}{An Application Programming Interface (API) is a particular set of rules and specifications that a software program can follow to access and make use of the services and resources provided by another particular software program that implements that API}

\let\textbfOld\textbf
\renewcommand{\textbf}[1]{#1}

\addAcronymAndGlossary{wcrt}
{WCRT}
{WCRTs}
{\protect\iflanguage{frenchb}{\textbf{P}ire \textbf{T}emps de \textbf{R}{\'e}ponse}{\textbf{W}orst \textbf{C}ase \textbf{R}esponse \textbf{T}ime}}
{\protect\iflanguage{frenchb}{\textbf{P}ires \textbf{T}emps de \textbf{R}{\'e}ponse}{\textbf{W}orst \textbf{C}ase \textbf{R}esponse \textbf{T}imes}}
{The \gls{wcrt} of a task is the maximum duration between the activation of the task and the moment it finishes its execution}

\addAcronymAndGlossary{wcet}{WCET}{WCETs}{\textbf{W}orst \textbf{C}ase \textbf{E}xecution \textbf{T}ime}{\textbf{W}orst \textbf{C}ase \textbf{E}xecution \textbf{T}imes}{The \gls{wcet} of a task is the maximum execution time required by the task to complete}

\addAcronymAndGlossary{rt}{\protect\iflanguage{frenchb}{TR}{RT}}{}{\protect\iflanguage{frenchb}{\textbf{T}emps-\textbf{R}{\'e}el}{\textbf{R}eal-\textbf{T}ime}}{}{}
\addAcronymAndGlossary{rtsj}{RTSJ}{}{\textbf{R}eal-\textbf{T}ime \textbf{S}pecification for \textbf{J}ava}{}{}

\addAcronymAndGlossary{stask}
{S-Task}
{S-Tasks}
{\protect\iflanguage{frenchb}{\textbf{T{\^a}che} \textbf{S}{\'e}quentielle}{\textbf{S}equential \textbf{Task}}}
{\protect\iflanguage{frenchb}{\textbf{T{\^a}ches} \textbf{S}{\'e}quentielles}{\textbf{S}equential \textbf{Tasks}}}
{Task model presented in Definition~\ref{def:sysmod_task_sequential_periodic} for the periodic case and Definition~\ref{def:sysmod_task_sequential_sporadic} for the sporadic case. (See page~\pageref{def:sysmod_task_sequential_periodic})}

\addAcronymAndGlossary{ptask}
{P-Task}
{P-Tasks}
{\protect\iflanguage{frenchb}{\textbf{T{\^a}che} \textbf{P}arall{\`e}le}{\textbf{P}arallel \textbf{Task}}}
{\protect\iflanguage{frenchb}{\textbf{T{\^a}ches} \textbf{P}arall{\`e}les}{\textbf{P}arallel \textbf{Tasks}}}
{Task model presented in Definition~\ref{def:sysmod_task_parallel_gang} (See page \pageref{def:sysmod_task_parallel_gang}) for the Gang model, Definition~\ref{def:sysmod_task_parallel_forkjoin} (See page \pageref{def:sysmod_task_parallel_forkjoin}) for the Fork-Join model and Definition~\ref{def:parallel_mpmt_taskmodel} (See page \pageref{def:parallel_mpmt_taskmodel}) for the \gls{mpmt} model}

\addAcronymAndGlossary{psched}{P-Scheduling}{}{\protect\iflanguage{frenchb}{\textbf{Ordonnancement} \textbf{P}artitionn{\'e}}{\textbf{P}artitioned \textbf{Scheduling}}}{}{}
\addAcronymAndGlossary{gsched}{G-Scheduling}{}{\protect\iflanguage{frenchb}{\textbf{Ordonnancement} \textbf{G}lobal}{\textbf{G}lobal \textbf{Scheduling}}}{}{}
\addAcronymAndGlossary{spsched}{SP-Scheduling}{}{\protect\iflanguage{frenchb}{\textbf{Ordonnancement} \textbf{S}emi-\textbf{P}artitionn{\'e}}{\textbf{S}emi-\textbf{P}artitioned \textbf{Scheduling}}}{}{}

\addAcronymAndGlossary{edf}{EDF}{}{\textbf{E}arliest \textbf{D}eadline \textbf{F}irst}{}{}
\addAcronymAndGlossary{llf}{LLF}{}{\textbf{L}east \textbf{L}axity \textbf{F}irst}{}{}
\addAcronymAndGlossary{dm}{DM}{}{\textbf{D}eadline \textbf{M}onotonic}{}{}
\addAcronymAndGlossary{rm}{RM}{}{\textbf{R}ate \textbf{M}onotonic}{}{}
\addAcronymAndGlossary{opa}{OPA}{}{\textbf{O}ptimal \textbf{P}riority \textbf{A}ssignment}{}{}
\addAcronymAndGlossary{im}{IM}{}{\textbf{I}ndex \textbf{M}onotonic}{}{}
\addAcronymAndGlossary{rma}{RMA}{}{\textbf{R}ate \textbf{M}onotonic \textbf{A}nalysis}{}{}

\addAcronymAndGlossary{ideadline}{I-Deadline}{I-Deadlines}{\textbf{I}mplicit \textbf{Deadline}}{\textbf{I}mplicit \textbf{Deadlines}}{A task is said to have \glsentrytext{ideadline} when its deadline is equal to its period, so $D_i = T_i$}
\addAcronymAndGlossary{cdeadline}{C-Deadline}{C-Deadlines}{\textbf{C}onstrained \textbf{Deadline}}{\textbf{C}onstraint \textbf{Deadlines}}{A task is said to have \glsentrytext{cdeadline} when its deadline is lower or equal to its period, so $D_i \leq T_i$}
\addAcronymAndGlossary{adeadline}{A-Deadline}{A-Deadlines}{\textbf{A}rbitraty \textbf{Deadline}}{\textbf{A}rbitrary \textbf{Deadlines}}{A task is said to have \glsentrytext{adeadline} when there is no link between its deadline and its period, so $D_i \leq T_i$ or $D_i \geq T_i$}

\addAcronymAndGlossary{ntest}{N-Test}{N-Tests}{\textbf{N}ecessary \textbf{Test}}{\textbf{N}ecessary \textbf{Tests}}{A test is said to be necessary if a negative result allows us to reject the proposition but a positive result does not allow us to accept the proposition. See Definition~\ref{def:analysis_ntest} and example on page \pageref{def:analysis_ntest}}
\addAcronymAndGlossary{stest}{S-Test}{S-Tests}{\textbf{S}ufficient \textbf{Test}}{\textbf{S}ufficient \textbf{Tests}}{A test is said to be sufficient if a positive result allows us to accept the proposition but a negative result does not allow us to reject the proposition. See Definition~\ref{def:analysis_stest} and example on page \pageref{def:analysis_stest}}
\addAcronymAndGlossary{nstest}{NS-Test}{NS-Tests}{\textbf{N}ecessary and \textbf{S}ufficient \textbf{Test}}{\textbf{N}ecessary and \textbf{S}ufficient \textbf{Tests}}{A test is said to be necessary and sufficient if a positive result allows us to accept the proposition and a negative result allows us to reject the proposition. See Definition~\ref{def:analysis_nstest} and example on page \pageref{def:analysis_nstest}}


\addAcronymAndGlossary{rbf}{RBF}{}{\textbf{R}equest \textbf{B}ound \textbf{F}unction}{}{The \gls{rbf} represents the upper bound of the work load generated by all tasks with activation instants included within the interval $[0;t)$. See example on page \pageref{eq:sysmod_task_sequential_tasksetrbf}}
\addAcronymAndGlossary{dbf}{DBF}{}{\textbf{D}emand \textbf{B}ound \textbf{F}unction}{}{The \gls{dbf} represents the upper bound of the work load generated by all tasks with activation instants and absolute deadlines within the interval $[0;t]$. See example on page \pageref{eq:sysmod_task_sequential_tasksetdbf}}


\addAcronymAndGlossary{restmig}
{Rest-Migration}
{Rest-Migrations}
{\protect\iflanguage{frenchb}{\textbf{Migration} \textbf{Rest}reinte}{\textbf{Rest}ricted \textbf{Migration}}}
{\protect\iflanguage{frenchb}{\textbf{Migrations} \textbf{Rest}reintes}{\textbf{Rest}ricted \textbf{Migrations}}}
{In the \gls{spsched} approach, \gls{restmig} refers to the case where migration is allowed, but only at job boundaries. A job is executed on one processor but successive jobs of a task can be executed on different processors. See Figure~\ref{fig:spsched_main_example_restmig} on page \pageref{fig:spsched_main_example_restmig}}
\addAcronymAndGlossary{unrestmig}
{UnRest-Migration}
{UnRest-Migrations}
{\protect\iflanguage{frenchb}{\textbf{Migration} \textbf{Non-Rest}reinte}{\textbf{UnRest}ricted \textbf{Migration}}}
{\protect\iflanguage{frenchb}{\textbf{Migrations} \textbf{Non-Rest}reintes}{\textbf{UnRest}ricted \textbf{Migrations}}}
{In the \gls{spsched} approach, \gls{unrestmig} refers to the case where migration is allowed, and a job can be portioned between multiple processors. A job can start its execution on one processor and complete on an other processor. See Figure~\ref{fig:spsched_main_example_unrestmig} on page \pageref{fig:spsched_main_example_unrestmig}}

\addAcronymAndGlossary{rrjm}{RRJM}{}{\textbf{R}ound-\textbf{R}obin \textbf{J}ob \textbf{M}igration}{}{Job placement heuristic used for the \gls{spsched} approach with \gls{restmig}. See Definition~\ref{def:spsched_restmig_roundrobin} on page \pageref{def:spsched_restmig_roundrobin}}
\addAcronymAndGlossary{mld}{MLD}{}{\textbf{M}igration at \textbf{L}ocal \textbf{D}eadline}{}{In the \gls{spsched} approach with \gls{unrestmig}, \gls{mld} refers to the solution of using local deadlines to specify migration points. See Definition~\ref{def:spsched_unrestmig_taskmodel} on page \pageref{def:spsched_unrestmig_taskmodel}}




\addAcronymAndGlossary{mpi}{MPI}{}{\textbf{M}essage \textbf{P}assing \textbf{I}nterface}{}{}
\addAcronymAndGlossary{openmp}{OpenMP}{}{\textbf{O}pen \textbf{M}ulti-\textbf{P}rocessing}{}{}
\addAcronymAndGlossary{pthread}{Pthread}{}{\textbf{P}OSIX \textbf{thread}}{}{}
\addAcronymAndGlossary{tbb}{TBB}{}{\textbf{T}hreading \textbf{B}uilding \textbf{B}locks}{}{}

\addAcronymAndGlossary{ftp}{FTP}{}{\textbf{F}ixed \textbf{T}ask \textbf{P}riority}{}{}
\addAcronymAndGlossary{dtp}{DTP}{}{\textbf{D}ynamic \textbf{T}ask \textbf{P}riority}{}{}
\addAcronymAndGlossary{fjp}{FJP}{}{\textbf{F}ixed \textbf{J}ob \textbf{P}riority}{}{}
\addAcronymAndGlossary{djp}{DJP}{}{\textbf{D}ynamic \textbf{J}ob \textbf{P}riority}{}{}
\addAcronymAndGlossary{fpp}{FPP}{}{\textbf{F}ixed \textbf{P}rocess \textbf{P}riority}{}{}
\addAcronymAndGlossary{dpp}{DPP}{}{\textbf{D}ynamic \textbf{P}rocess \textbf{P}riority}{}{}
\addAcronymAndGlossary{fsp}{FSP}{}{\textbf{F}ixed \textbf{S}ub-program \textbf{P}riority}{}{}
\addAcronymAndGlossary{fthp}{FThP}{}{\textbf{F}ixed \textbf{Th}read \textbf{P}riority}{}{}
\addAcronymAndGlossary{dthp}{DThP}{}{\textbf{D}ynamic \textbf{Th}read \textbf{P}riority}{}{}
\addAcronymAndGlossary{lsf}{LSF}{}{\textbf{L}ongest \textbf{S}ub-program \textbf{F}irst}{}{}

\addAcronymAndGlossary{ci}{CI}{}{\textbf{C}arry \textbf{I}n}{}{A \gls{ci} task refers to a task with one job with arrival instant earlier than the interval $[a;b]$ and deadline in the interval $[a;b]$. See Figure~\ref{fig:ptask_mpmt_wcrt_sequentialtaskCI} on page \pageref{fig:ptask_mpmt_wcrt_sequentialtaskCI}}
\addAcronymAndGlossary{nc}{NC}{}{\textbf{N}on \textbf{C}arry-in}{}{A \gls{nc} task is the opposite of a \gls{ci} task. It refers to a task with one job with arrival instant and deadline in the interval $[a;b]$. See Figure~\ref{fig:ptask_mpmt_wcrt_sequentialtaskNC} on page \pageref{fig:ptask_mpmt_wcrt_sequentialtaskNC}}

\addAcronymAndGlossary{mpmt}{MPMT}{}{\textbf{M}ulti-\textbf{P}hase \textbf{M}ulti-\textbf{T}hread}{}{\gls{ptask} model given by Definition~\ref{def:parallel_mpmt_taskmodel} on page \pageref{def:parallel_mpmt_taskmodel}}


\addAcronymAndGlossary
{fortas}
{FORTAS}
{}
%{\protect\iflanguage{frenchb}{\textbf{F}ramework p\textbf{O}ur l'\textbf{A}nalyse et la \textbf{S}imulation de syst{\`e}mes \textbf{T}emps \textbf{R}{\'e}el}{\textbf{F}ramework f\textbf{O}r \textbf{R}eal-\textbf{T}ime \textbf{A}nalysis and \textbf{S}imulation}}
{\textbf{F}ramework f\textbf{O}r \textbf{R}eal-\textbf{T}ime \textbf{A}nalysis and \textbf{S}imulation}
{}
{}
{}

\addAcronymAndGlossary{gui}{GUI}{}{\textbf{G}raphical \textbf{U}ser \textbf{I}nterface}{}{}

\addAcronymAndGlossary{uml}{UML}{}{\textbf{U}nified \textbf{M}odeling \textbf{L}anguage}{}{}
\addAcronymAndGlossary{ebsf}{EBSF}{}{\textbf{E}uropean \textbf{B}us \textbf{S}ystem of the \textbf{F}uture}{}{\gls{ebsf} is an initiative of the European Commission under the Seventh Framework Programme for Research and Technological Development. Starting in September 2008; \gls{ebsf} is a four-year project with an overall budget of 26 million Euros (16 millions cofunded) and is coordinated by UITP, the International Association of Public Transport. See \url{http://www.ebsf.eu/}}
\addAcronymAndGlossary{emma}{EMMA}{}{\textbf{E}nvironment \textbf{M}onitoring and \textbf{M}anagement \textbf{A}gents}{}{}


\renewcommand{\textbf}[1]{\textbfOld{#1}}

